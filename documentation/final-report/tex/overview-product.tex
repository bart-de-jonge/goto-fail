\section{Overview of the developed and implemented software product}

Our software product consists of three main components. The first one is the scripting application which is used to create a script before the repetitions even start. The second component is the web application. This part is used during the actual recording, and provides all the involved people with a view that displays information that is useful for them. The final part is the camera control center. This is used to create presets that can later be recalled, either manually or automatically. We'll discuss each of these components separately, followed by a section on how these components work together.

\subsection{The scripting application}
The general idea of the scripting application is that the director can easily create scripts by creating shots, giving them a name and description, and moving them around, making them shorter or longer until he is happy with the result. In our scripting application, there is also collision detection, which means that if two shots for one camera overlap or are too close together, the shots will be highlighted to indicate a conflict. What the director can also do is create a so called "directorshot", in which he can indicate which cameras should be used to take a certain shot. After that, the shots for the separate cameras can be automatically generated. A more detailed description of the features can be found in chapter 4.

\subsection{The web application}
In the web application, there are different views for people with different roles. The score caller can advance the count, either manually by tapping a button, or automatically by setting a time per count. This will be directly reflected in the view for the camera operators, in which they see the timelines of the cameras that they are controlling. They will see a line advancing through the timeline so they can easily see what shots they have to take in the nearby future. There is also a view for the director, in which he can see an overview of all the timelines and make last minute changes if needed. 

\subsection{The camera control center}
In the camera control center, it is possible to move the camera around and create presets. These presets can be recalled here as well. There is not really anything else to do here, but the presets will be used in the web application as well, as explained in the next section

\subsection{The complete system}
Of course, the different components work together. After a script has been created, it can be saved in XML format and sent to a webserver with one click on a button. The webserver accepts the file and initializes the live views with it. At this point, a camera operator could go to the web app, select his name and check which timelines he operates. These timelines will the be showed, containing the shots that were created in the scripting application.\\
It is then also possible to add certain presets to certain shots. After a preset has been created in the camera control center, it can be linked to a shot in the web application. Then, when the live production starts, our application makes sure that all the presets are loaded on time, so camera operators don't have to focus on recalling the presets, but can rather focus on making small adjustments to create the perfect shots.