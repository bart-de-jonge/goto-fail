\section{Overview of the developed and implemented software product}

Our software product consists of two main components. The first one is the scripting application which is used to create a script before the repetitions even start. The second component is the web application. This part is used during the actual recording, and provides all the involved people with a view that displays information that is useful for them.\\
 We have decided to split these components because they will be used in a completely different way. The scripting application is designed to be used on a desktop or laptop. It will only be used by a single person at the same time. Therefore, a Java application is a better choice performance-wise. Also, implementing the dragging and dropping functionality is easier to do in a Java application than in a web app. The web application will be used by multiple people at the same time who all work on different devices and screen sizes. Therefore, we needed this component to be accessible from multiple locations simultaneously, with a design responsive to screen size changes. A web application fits perfectly here, which is why we chose to implement the live view in such a web app.\\
We will discuss each of these components separately, followed by a section on how these components work together.

\subsection{The scripting application}
The general idea of the scripting application is that the director can easily create scripts by creating shots, giving them a name and description, and moving them around, making them shorter or longer until he is happy with the result. In our scripting application, there is also collision detection, which means that if two shots for one camera overlap or are too close together, the shots will be highlighted to indicate a conflict. What the director can also do is create a so called "directorshot", in which he can indicate which cameras should be used to take a certain shot. After that, the shots for the separate cameras can be automatically generated. A more detailed description of the features can be found in chapter 4.

\subsection{The web application}
In the web application, there are different views for people with different roles. The score caller can advance the count by tapping a button. This will be directly reflected in the view for the camera operators, in which they see the timelines of the cameras that they are controlling. They will see a line advancing through the timeline so they can easily see what shots they have to take in the nearby future. There is also a view for the director, in which he can see an overview of all the timelines and make last minute changes if needed. The director can start the live mode when the concert begins.

\subsection{The complete system}
Of course, the different components work together. After a script has been created, it can be saved in XML format and sent to the webserver with one click on a button. The webserver accepts the file and initializes the live views with it. At this point, a camera operator could go to the web app, select his name and check which timelines he operates. These timelines will the be showed, containing the shots that were created in the scripting application.\\
