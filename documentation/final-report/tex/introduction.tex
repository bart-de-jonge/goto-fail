\section{Introduction}

During the past 9 weeks, we have worked on designing, implementing and testing a software application. In this report, we will discuss the product and the process of developing it. We will start by explaining the problem that we have tried to solve, and the end-user requirements that we had to take into account.

\subsection{Problem description}
The software applicaton we have developed was focused on helping PolyCast, a classical recording company with clients world-wide\footnote{\url{http://www.polycast.nl/over}}. PolyCast does a lot of classical concert recordings, but this was stil a very manual operation. It involves paper scripts, manual camera operation and a lot of shouting and chaos during a recording. Because of this, camera operators cannot be as efficient as they could be when a more automated work flow would be used. This is where we stepped in, and created a software system that focuses on faster, digital script creation that can be linked with a web application, in which camera operators, score callers and directors all have their own view for during live production. We will explain these roles in more detail later. We have also integrated automatic preset recalling, so that camera operators can focus more on what is happening now, rather than what is coming later.

\subsection{End-user requirements}
The main requirement for the employees at PolyCast is that the software application should be easy and fast to use. Especially for the part of the application used during live recordings, there is not a lot of time, so the application should be automated as much as possible, with only few user interactions needed.\\
For the script-creating part of our application, there were a few requirements set by the people of PolyCast. The timeline orientation should be vertical. Also, it should be easy to create shots, move them around and resize them. Next to that, they would like it very much if there would be some kind of collision detection so they can see when an invalid script was created.

\subsection{Outline}
The remainder of this report is structured as follows: Chapter 2 gives an overview of the software product. In chapter 3, we reflect on the product and process. Chapter 4 contains a description of the developed functionalities. We discuss the Interaction Design part in chapter 5. The evaluation of the functional modules, the product and the failure analysis can be found in chapter 6. Chapter 7 contains the outlook.