\section*{Appendix B: User test instructions}
\begin{Large}
\textbf{User testing instructions - for observers}\\\\
\end{Large}
The purpose of this document is to shortly instruct you how the user testing is (preferably) done.\\\\
\begin{large}
\textbf{User testing preparations}\\\\
\end{large}
The user testing will be done using the \textbf{think aloud protocol}. This means you, as an observer, \textbf{should keep your thoughts to yourself}. The user will attempt to perform a set of tasks, and is supposed to do so without any interference, unless they specifically ask for help (when stuck), or when bugs/errors appear, or when they share too little information. Your job, as an observer, is two-fold.\\
Firstly, your job is to ensure users know what is expected of them. Instruct them on how testing will be done, if necessary. While they are testing, make sure that they keep talking.\\
Secondly, your job is to record what the user does. Please take extensive notes on the following:\\
\begin{enumerate}
\item What the user says.
\item What the user does.
\item Where the user encounters difficulties.
\end{enumerate}
Please ensure that you do not attempt to \textbf{interpret} a user\textquotesingle s actions and/or words. It is critical that you record everything in the most literal way possible. Evaluation can come afterwards.
\newpage

\begin{Large}
\textbf{User testing instructions - for participants}\\\\
\end{Large}
The purpose of this document is to shortly introduce you to our application, and instruct you to how the user testing is (preferably) done. We\textquotesingle d like to thank you for your participation, as this will be of great use to us.\\\\
\begin{large}
\textbf{Brief introduction to applications}\\
\end{large}
We have two sets of applications, focused on two components of Polycast\textquotesingle s workflow. We have a scripting application, used by the director, and a web application, which is used by both the director, camera operators and the shot caller, during the live performance. The scripting application can be used by the director to prepare and create camera scripts. The web application serves as a live \"view\" of script data for all users, showing them what they (in their function) are required to see, or are allowed to do.\\\\
\begin{large}
\textbf{User testing preparations}\\\\
\end{large}
The user testing will be done using the \textbf{think aloud protocol}. This means we\textquotesingle d like to invite you to, as much as possible, say whatever comes to mind while you work on tasks. \textbf{Anything goes}. So this definitely includes:\\
\begin{itemize}
\item What you are looking at (specifically)....
\item What are you thinking...
\item What are you doing...(or what you think you are doing!)
\item What you are feeling...
\end{itemize}
Your role (as tester) will be two-fold. At first, we\textquotesingle d like you to look from a Director\textquotesingle s perspective, and we\textquotesingle ll have you perform some tasks in the scripting application. Afterwards, we\textquotesingle ll have you perform some tasks on the web application, as if you were either a generic user, a camera operator, or the shot caller.\\\\
We\textquotesingle ve prepared a list of tasks we would like you to complete, one by one, on the back of this page. If you get stuck on a task, \textbf{don\textquotesingle t sweat it}. You can always ask the observers for help. They are instructed to not help you if you get stuck, unless you specifically ask them to. They are also instructed to not interact with you in any other way, so that is all they can do for you.\\\\
Good luck, and thank you!\\\\\\
\begin{large}
\textbf{Scripting application tasks}\\
\end{large}
The scripting application should be up and running when you sit down. Below is a list of tasks you can attempt to do, in order. Remember, if you get stuck, ask an observer for help!\\
\begin{enumerate}
\item Create a new project. Make sure the project has some instruments, at least one camera type, and a multitude of cameras set up.
\item Add a new camera shot to one of the cameras you created
\item Move the camera shot to another camera.
\item Make the camera shot start later/earlier, and end later/earlier
\item Modify the name of the camera shot.
\item Add or remove some instruments to the camera shot
\item Add another camera shot to another camera
\item Make the camera shots collide with one another.
\item Un-collide the camera shots.
\item Add a director shot. Make sure it uses a number of cameras, at least.
\item Generate camera shots from this director shot.
\item Make the director shot use more/less cameras.
\item Save your project.
\item Change color preferences of the application to another color.
\item Upload the project to the webserver.
\item Go to the website.
\end{enumerate}
Now please continue with the Web application tasks below.\\\\\\
\begin{large}
\textbf{Web application tasks}\\
\end{large}
The website should now be up and running. Below is a list of tasks you can attempt to do, in order. Note that you\textquotesingle ll be acting as at least two kinds of users: a camera operator and a shot caller.\\
\begin{enumerate}
\item Go to the user selection view.
\item Add yourself as a user. Make sure you are \textbf{a shot caller}.
\item Select yourself, and continue.
\item Go to the timeline view. Observe the camera shots you placed a while ago.
\item Attempt to move the score counter, using the shot caller view.
\item Once you\textquotesingle re done, delete yourself as a user.
\end{enumerate}
Congratulations, you\textquotesingle re done. Thank you for your assistance. Feel free to discuss any issues or give your thoughts on the entire application, to the observers.
