\section{Outlook}

In this section, we will explore possible future improvements and the strategy that could be used to achieve those.\\\\
We believe the highest potential for improvement lies in the camera presets. At the moment, the interface to move the camera and create presets is functional, but very basic. This can be improved greatly, with better visualization of the saved presets. Also, a better interface for moving the cameras around should be provided.\\\\
There were also some requirements that we did not get to, as specified in the previous chapter. These are, of course, also possible improvements. A major feature that could be added is the editing of the timeline during the recording by the director, and the propagation of these changes to the camera operators.\\
Something else that needs further attention is the current fixed timeline size in the scripting application. Creating very long scripts is problematic now. A feature that could be added is automatic extension of the timeline length when scrolling down.

\subsection{Cooperation with BeNine}
Together with another group in our context, BeNine\footnote{\url{https://github.com/Kingdorian/Contextproject-BeNine}}, we have decided that a possible improvement would be to integrate our applications together. They have especially focused on the preset/camera movement part which is exactly our weak spot at the moment. We have already explored the feasibility of this integration, and we found that this should be quite easy to do. The effect would be that presets are created using their application, and that they would be used in our web application by automatically recalling them when needed. This would create a software system that is really complete and can be used throughout the whole process of a classical recording.


