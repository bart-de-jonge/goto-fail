\section{Outlook}

In this section, we will explore possible future improvements and the strategy that could be used to achieve those.\\\\

\subsection{Cooperation with BeNine}
As already indicated in the previous chapter, we will be integrating our product with that of BeNine\footnote{\url{https://github.com/Kingdorian/Contextproject-BeNine}}. It is put under this future work section because it will not be finished before the official deadline, but the initial integration of our systems will actually be done next week already.\\
The idea is that the camera presets will be created in the application of BeNine. Then, in our web application, we will make it possible to link these presets to camera shots. During the live production, these presets will be automatically recalled, making the lives of the camera operators a lot easier because they do not have to look ahead all of the time, but can rather focus on what is going on in the present.\\
This integration would create a software system that is complete and can be used throughout the whole process of a classical music recording.\\\\
A long-term future improvement related to this is integrating the two systems more flawlessly. For example, we would style the applications similarly, and make them use the same webserver so that the user does not have to run multiple servers and visit multiple websites.

\subsection{Other improvements}
There were also some requirements in our core application that we did not get to, as specified in the previous chapter. These are, of course, also possible improvements. A major feature that could be added is the editing of the timeline during the recording by the director, and the propagation of these changes to the camera operators.\\
Something else that needs further attention is the current fixed timeline size in the scripting application. Creating very long scripts is a problematic task. A feature that could be added is automatic extension of the timeline length when scrolling down.



